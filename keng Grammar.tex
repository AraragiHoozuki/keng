%!TEX program = xelatex
\documentclass[UTF8]{ctexbook}

\usepackage{fontspec}
\setmainfont[Mapping=tex -text]{Palemonas}

\title{Keng go go ma}
\author{Rholin}
\date{\today}
\begin{document}
\maketitle

\section{前言}
這是一篇坑語 (keng go) 的語法,大概在有生之年能夠寫完。按照我的預想,坑語應該是一個有那麼一點文言風格的分析語。不過現在看來,其發展路線已經遠遠偏離了我的本意……

%-------------------------------------------------------------------------------------------------------------------------------------------------------------------
\chapter{音系}
\section{元音}
坑語一共有 6 個單元音,分別是 a [a], e [e], i [i], o [o], u [u], y [y]. 單元音不分長短。此外 i, u 可以作 a, e 的韻尾 (ai, au, ei, eu); i, u, y 可以作任何元音的介音。

\section{輔音}
脣:p [b], b [b], w [w], m [m] \par
齒/齒齦:t [t], d [d], n [n], r [r], l [l], s [s], z [z] \par
捲舌:ṭ [ʈ], ḍ [ɖ], ṇ [ɳ], ḷ [ɭ] \par
硬齶:j [j] \par
軟齶:k [k], g [g], ng [ŋ] \par

\section{音節結構}
(s)C(C)(V)V(V/C)(s)(ḷ), 具體是什麼意思我就不說了,反正說了也沒用,以後還是我怎麼高興怎麼來。

%------------------------------------------------------------------------------------------------------------------------------------------------------------------
\chapter{詞法}
\section{構詞法}
\subsection{派生/轉化}
增加介音 [i](名/形 > 動):\textit{a}「存在之物、存在」> \textit{ja}「是」\par
g-j- 構成動詞反復式:\textit{jet}「吃」> \textit{gjjet}「吃藥」、\textit{pjut}「切開,剪開」> \textit{gpjjut}「裁剪」\par
s- 構成動詞/形容詞使動:\textit{maf}「動」> \textit{smaf}「感動、使變化」、\textit{griz}「返回」> \textit{sgriz}「遣返」、\textit{alg}「快樂」> \textit{salg}「使快樂」 \par
弱化-r 構成動作的結果:\textit{pjut}「切」> \textit{bjutr}「片」、\textit{brjo}「死」> \textit{wrjor}「屍」\par
轉化(不改變詞形的情況下詞義或詞性改變):\textit{leng}「工作(動)」> \textit{leng}「工人」、\textit{kal}「機會」> \textit{kal}「投機」\par

\subsection{合成}
名+動,表示執行該動作者:\textit{pmeut-pjut}「帆」(風+切)、\textit{ka-pjut}「寒風」(花+切)。實際使用中連字符可以省略。\textit{nam-du}「牛逼的人」(天+干) \par
形+名:\textit{bal-rang}「祭司」(白+手)、\textit{wnau-hvat}「傻逼」(愚+魚)\par
名+名:\textit{nam-omp}「天目」(天+目)、\textit{pmeut-put}「風刃(大概是一種低級法術的名字?)」(風+刃)\par


\end{document}